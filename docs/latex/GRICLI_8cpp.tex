\hypertarget{GRICLI_8cpp}{\section{framework/source/\-G\-R\-I\-C\-L\-I.cpp \-File \-Reference}
\label{GRICLI_8cpp}\index{framework/source/\-G\-R\-I\-C\-L\-I.\-cpp@{framework/source/\-G\-R\-I\-C\-L\-I.\-cpp}}
}
{\ttfamily \#include \char`\"{}iostream\char`\"{}}\*
{\ttfamily \#include $<$\-Q\-List$>$}\*
{\ttfamily \#include $<$\-Q\-String\-List$>$}\*
{\ttfamily \#include \char`\"{}\-G\-R\-I\-C\-L\-I.\-h\char`\"{}}\*
